% Options for packages loaded elsewhere
\PassOptionsToPackage{unicode}{hyperref}
\PassOptionsToPackage{hyphens}{url}
%
\documentclass[
  ignorenonframetext,
]{beamer}
\usepackage{pgfpages}
\setbeamertemplate{caption}[numbered]
\setbeamertemplate{caption label separator}{: }
\setbeamercolor{caption name}{fg=normal text.fg}
\beamertemplatenavigationsymbolsempty
% Prevent slide breaks in the middle of a paragraph
\widowpenalties 1 10000
\raggedbottom
\setbeamertemplate{part page}{
  \centering
  \begin{beamercolorbox}[sep=16pt,center]{part title}
    \usebeamerfont{part title}\insertpart\par
  \end{beamercolorbox}
}
\setbeamertemplate{section page}{
  \centering
  \begin{beamercolorbox}[sep=12pt,center]{part title}
    \usebeamerfont{section title}\insertsection\par
  \end{beamercolorbox}
}
\setbeamertemplate{subsection page}{
  \centering
  \begin{beamercolorbox}[sep=8pt,center]{part title}
    \usebeamerfont{subsection title}\insertsubsection\par
  \end{beamercolorbox}
}
\AtBeginPart{
  \frame{\partpage}
}
\AtBeginSection{
  \ifbibliography
  \else
    \frame{\sectionpage}
  \fi
}
\AtBeginSubsection{
  \frame{\subsectionpage}
}
\usepackage{amsmath,amssymb}
\usepackage{iftex}
\ifPDFTeX
  \usepackage[T1]{fontenc}
  \usepackage[utf8]{inputenc}
  \usepackage{textcomp} % provide euro and other symbols
\else % if luatex or xetex
  \usepackage{unicode-math} % this also loads fontspec
  \defaultfontfeatures{Scale=MatchLowercase}
  \defaultfontfeatures[\rmfamily]{Ligatures=TeX,Scale=1}
\fi
\usepackage{lmodern}
\ifPDFTeX\else
  % xetex/luatex font selection
\fi
% Use upquote if available, for straight quotes in verbatim environments
\IfFileExists{upquote.sty}{\usepackage{upquote}}{}
\IfFileExists{microtype.sty}{% use microtype if available
  \usepackage[]{microtype}
  \UseMicrotypeSet[protrusion]{basicmath} % disable protrusion for tt fonts
}{}
\makeatletter
\@ifundefined{KOMAClassName}{% if non-KOMA class
  \IfFileExists{parskip.sty}{%
    \usepackage{parskip}
  }{% else
    \setlength{\parindent}{0pt}
    \setlength{\parskip}{6pt plus 2pt minus 1pt}}
}{% if KOMA class
  \KOMAoptions{parskip=half}}
\makeatother
\usepackage{xcolor}
\newif\ifbibliography
\usepackage{color}
\usepackage{fancyvrb}
\newcommand{\VerbBar}{|}
\newcommand{\VERB}{\Verb[commandchars=\\\{\}]}
\DefineVerbatimEnvironment{Highlighting}{Verbatim}{commandchars=\\\{\}}
% Add ',fontsize=\small' for more characters per line
\usepackage{framed}
\definecolor{shadecolor}{RGB}{248,248,248}
\newenvironment{Shaded}{\begin{snugshade}}{\end{snugshade}}
\newcommand{\AlertTok}[1]{\textcolor[rgb]{0.94,0.16,0.16}{#1}}
\newcommand{\AnnotationTok}[1]{\textcolor[rgb]{0.56,0.35,0.01}{\textbf{\textit{#1}}}}
\newcommand{\AttributeTok}[1]{\textcolor[rgb]{0.13,0.29,0.53}{#1}}
\newcommand{\BaseNTok}[1]{\textcolor[rgb]{0.00,0.00,0.81}{#1}}
\newcommand{\BuiltInTok}[1]{#1}
\newcommand{\CharTok}[1]{\textcolor[rgb]{0.31,0.60,0.02}{#1}}
\newcommand{\CommentTok}[1]{\textcolor[rgb]{0.56,0.35,0.01}{\textit{#1}}}
\newcommand{\CommentVarTok}[1]{\textcolor[rgb]{0.56,0.35,0.01}{\textbf{\textit{#1}}}}
\newcommand{\ConstantTok}[1]{\textcolor[rgb]{0.56,0.35,0.01}{#1}}
\newcommand{\ControlFlowTok}[1]{\textcolor[rgb]{0.13,0.29,0.53}{\textbf{#1}}}
\newcommand{\DataTypeTok}[1]{\textcolor[rgb]{0.13,0.29,0.53}{#1}}
\newcommand{\DecValTok}[1]{\textcolor[rgb]{0.00,0.00,0.81}{#1}}
\newcommand{\DocumentationTok}[1]{\textcolor[rgb]{0.56,0.35,0.01}{\textbf{\textit{#1}}}}
\newcommand{\ErrorTok}[1]{\textcolor[rgb]{0.64,0.00,0.00}{\textbf{#1}}}
\newcommand{\ExtensionTok}[1]{#1}
\newcommand{\FloatTok}[1]{\textcolor[rgb]{0.00,0.00,0.81}{#1}}
\newcommand{\FunctionTok}[1]{\textcolor[rgb]{0.13,0.29,0.53}{\textbf{#1}}}
\newcommand{\ImportTok}[1]{#1}
\newcommand{\InformationTok}[1]{\textcolor[rgb]{0.56,0.35,0.01}{\textbf{\textit{#1}}}}
\newcommand{\KeywordTok}[1]{\textcolor[rgb]{0.13,0.29,0.53}{\textbf{#1}}}
\newcommand{\NormalTok}[1]{#1}
\newcommand{\OperatorTok}[1]{\textcolor[rgb]{0.81,0.36,0.00}{\textbf{#1}}}
\newcommand{\OtherTok}[1]{\textcolor[rgb]{0.56,0.35,0.01}{#1}}
\newcommand{\PreprocessorTok}[1]{\textcolor[rgb]{0.56,0.35,0.01}{\textit{#1}}}
\newcommand{\RegionMarkerTok}[1]{#1}
\newcommand{\SpecialCharTok}[1]{\textcolor[rgb]{0.81,0.36,0.00}{\textbf{#1}}}
\newcommand{\SpecialStringTok}[1]{\textcolor[rgb]{0.31,0.60,0.02}{#1}}
\newcommand{\StringTok}[1]{\textcolor[rgb]{0.31,0.60,0.02}{#1}}
\newcommand{\VariableTok}[1]{\textcolor[rgb]{0.00,0.00,0.00}{#1}}
\newcommand{\VerbatimStringTok}[1]{\textcolor[rgb]{0.31,0.60,0.02}{#1}}
\newcommand{\WarningTok}[1]{\textcolor[rgb]{0.56,0.35,0.01}{\textbf{\textit{#1}}}}
\usepackage{graphicx}
\makeatletter
\def\maxwidth{\ifdim\Gin@nat@width>\linewidth\linewidth\else\Gin@nat@width\fi}
\def\maxheight{\ifdim\Gin@nat@height>\textheight\textheight\else\Gin@nat@height\fi}
\makeatother
% Scale images if necessary, so that they will not overflow the page
% margins by default, and it is still possible to overwrite the defaults
% using explicit options in \includegraphics[width, height, ...]{}
\setkeys{Gin}{width=\maxwidth,height=\maxheight,keepaspectratio}
% Set default figure placement to htbp
\makeatletter
\def\fps@figure{htbp}
\makeatother
\setlength{\emergencystretch}{3em} % prevent overfull lines
\providecommand{\tightlist}{%
  \setlength{\itemsep}{0pt}\setlength{\parskip}{0pt}}
\setcounter{secnumdepth}{-\maxdimen} % remove section numbering
\ifLuaTeX
  \usepackage{selnolig}  % disable illegal ligatures
\fi
\usepackage{bookmark}
\IfFileExists{xurl.sty}{\usepackage{xurl}}{} % add URL line breaks if available
\urlstyle{same}
\hypersetup{
  pdftitle={regression\_analysis\_prageeth},
  pdfauthor={my author},
  hidelinks,
  pdfcreator={LaTeX via pandoc}}

\title{regression\_analysis\_prageeth}
\author{my author}
\date{2025-02-25}

\begin{document}
\frame{\titlepage}

\begin{frame}[fragile]{}
\phantomsection\label{section}
\#\texttt{\{r\ setup,\ include=FALSE\}\ \#knitr::opts\_chunk\$set(echo\ =\ FALSE)\ \#}

\begin{block}{R Markdown}
\phantomsection\label{r-markdown}
This is an R Markdown presentation. Markdown is a simple formatting
syntax for authoring HTML, PDF, and MS Word documents. For more details
on using R Markdown see \url{http://rmarkdown.rstudio.com}.

When you click the \textbf{Knit} button a document will be generated
that includes both content as well as the output of any embedded R code
chunks within the document.
\end{block}

\begin{block}{Slide with Bullets}
\phantomsection\label{slide-with-bullets}
\begin{itemize}
\tightlist
\item
  Bullet 1
\item
  Bullet 2
\item
  Bullet 3
\end{itemize}
\end{block}

\begin{block}{Slide with R Output}
\phantomsection\label{slide-with-r-output}
\begin{Shaded}
\begin{Highlighting}[]
\FunctionTok{summary}\NormalTok{(cars)}
\end{Highlighting}
\end{Shaded}

\begin{verbatim}
##      speed           dist       
##  Min.   : 4.0   Min.   :  2.00  
##  1st Qu.:12.0   1st Qu.: 26.00  
##  Median :15.0   Median : 36.00  
##  Mean   :15.4   Mean   : 42.98  
##  3rd Qu.:19.0   3rd Qu.: 56.00  
##  Max.   :25.0   Max.   :120.00
\end{verbatim}
\end{block}

\begin{block}{Slide with Plot}
\phantomsection\label{slide-with-plot}
\begin{Shaded}
\begin{Highlighting}[]
\FunctionTok{plot}\NormalTok{(pressure)}
\end{Highlighting}
\end{Shaded}

\includegraphics{myRegression_files/figure-beamer/pressure-1.pdf}
\end{block}
\end{frame}

\begin{frame}[fragile]{Start my work.}
\phantomsection\label{start-my-work.}
\begin{block}{}
\phantomsection\label{section-1}
\begin{Shaded}
\begin{Highlighting}[]
\NormalTok{house\_df }\OtherTok{\textless{}{-}} \FunctionTok{read.csv}\NormalTok{(}\StringTok{"C:}\SpecialCharTok{\textbackslash{}\textbackslash{}}\StringTok{Users}\SpecialCharTok{\textbackslash{}\textbackslash{}}\StringTok{Prageeth}\SpecialCharTok{\textbackslash{}\textbackslash{}}\StringTok{Source}\SpecialCharTok{\textbackslash{}\textbackslash{}}\StringTok{MSC}\SpecialCharTok{\textbackslash{}\textbackslash{}}\StringTok{seminar{-}course}\SpecialCharTok{\textbackslash{}\textbackslash{}}\StringTok{advanced{-}regression{-}kc}\SpecialCharTok{\textbackslash{}\textbackslash{}}\StringTok{kc\_house\_data.csv"}\NormalTok{, }\AttributeTok{stringsAsFactors =} \ConstantTok{TRUE}\NormalTok{)}
\NormalTok{features }\OtherTok{\textless{}{-}} \FunctionTok{c}\NormalTok{(}\StringTok{"bedrooms"}\NormalTok{, }\StringTok{"bathrooms"}\NormalTok{, }\StringTok{"sqft\_living"}\NormalTok{, }\StringTok{"sqft\_lot"}\NormalTok{, }\StringTok{"floors"}\NormalTok{, }
              \StringTok{"condition"}\NormalTok{, }\StringTok{"grade"}\NormalTok{, }\StringTok{"sqft\_above"}\NormalTok{, }\StringTok{"sqft\_basement"}\NormalTok{, }\StringTok{"yr\_built"}\NormalTok{)}
\NormalTok{target }\OtherTok{\textless{}{-}} \StringTok{"price"}
\end{Highlighting}
\end{Shaded}
\end{block}
\end{frame}

\begin{frame}[fragile]{Simple Linear Regression}
\phantomsection\label{simple-linear-regression}
\begin{block}{}
\phantomsection\label{section-2}
\begin{Shaded}
\begin{Highlighting}[]
\NormalTok{simple\_lm }\OtherTok{\textless{}{-}} \FunctionTok{lm}\NormalTok{ ( price }\SpecialCharTok{\textasciitilde{}}\NormalTok{ sqft\_living ,}
                  \AttributeTok{data =}\NormalTok{ house\_df )}
\FunctionTok{ggplot}\NormalTok{(house\_df, }\FunctionTok{aes}\NormalTok{(}\AttributeTok{x =}\NormalTok{ sqft\_living, }\AttributeTok{y =}\NormalTok{ price)) }\SpecialCharTok{+}
  \FunctionTok{geom\_point}\NormalTok{(}\AttributeTok{alpha =} \FloatTok{0.5}\NormalTok{) }\SpecialCharTok{+}  \CommentTok{\# Scatter plot}
  \FunctionTok{geom\_smooth}\NormalTok{(}\AttributeTok{method =} \StringTok{"lm"}\NormalTok{, }\AttributeTok{color =} \StringTok{"blue"}\NormalTok{, }\AttributeTok{se =} \ConstantTok{FALSE}\NormalTok{) }\SpecialCharTok{+}\CommentTok{\# Regression line}
  \FunctionTok{labs}\NormalTok{(}\AttributeTok{title =} \StringTok{"Price vs. Size Fit"}\NormalTok{,}
       \AttributeTok{x =} \StringTok{"Size (sqft\_living)"}\NormalTok{,}
       \AttributeTok{y =} \StringTok{"Price"}\NormalTok{) }\SpecialCharTok{+}
  \FunctionTok{theme\_minimal}\NormalTok{()}
  
\NormalTok{simple\_lm}
\end{Highlighting}
\end{Shaded}
\end{block}
\end{frame}

\begin{frame}[fragile]{}
\phantomsection\label{section-3}
\begin{block}{output}
\phantomsection\label{output}
\includegraphics{myRegression_files/figure-beamer/unnamed-chunk-3-1.pdf}

\begin{verbatim}
## 
## Call:
## lm(formula = price ~ sqft_living, data = house_df)
## 
## Coefficients:
## (Intercept)  sqft_living  
##    -43580.7        280.6
\end{verbatim}
\end{block}
\end{frame}

\begin{frame}[fragile]{}
\phantomsection\label{section-4}
\#\#Model Summary (Extracts Coefficients, p-values, R²)

\begin{verbatim}
## 
## Call:
## lm(formula = price ~ sqft_living + sqft_lot + bathrooms + grade, 
##     data = house_df, na.action = na.omit)
## 
## Residuals:
##      Min       1Q   Median       3Q      Max 
## -1011695  -136513   -23045   100989  4782979 
## 
## Coefficients:
##               Estimate Std. Error t value Pr(>|t|)    
## (Intercept) -5.957e+05  1.325e+04 -44.950  < 2e-16 ***
## sqft_living  2.065e+02  3.364e+00  61.373  < 2e-16 ***
## sqft_lot    -2.664e-01  4.171e-02  -6.388 1.71e-10 ***
## bathrooms   -3.944e+04  3.443e+03 -11.456  < 2e-16 ***
## grade        1.037e+05  2.285e+03  45.379  < 2e-16 ***
## ---
## Signif. codes:  0 '***' 0.001 '**' 0.01 '*' 0.05 '.' 0.1 ' ' 1
## 
## Residual standard error: 249600 on 21608 degrees of freedom
## Multiple R-squared:  0.538,  Adjusted R-squared:  0.5379 
## F-statistic:  6291 on 4 and 21608 DF,  p-value: < 2.2e-16
\end{verbatim}
\end{frame}

\begin{frame}{}
\phantomsection\label{section-5}
\end{frame}

\begin{frame}{Extract RMSE}
\phantomsection\label{extract-rmse}
\end{frame}

\begin{frame}{Extract R-squared}
\phantomsection\label{extract-r-squared}
\end{frame}

\begin{frame}{Extract P-values}
\phantomsection\label{extract-p-values}
\end{frame}

\begin{frame}[fragile]{Print Results}
\phantomsection\label{print-results}
\begin{verbatim}
## RMSE: 249532.2
\end{verbatim}

\begin{verbatim}
## R²: 0.5380018
\end{verbatim}

\begin{verbatim}
## P-values:
\end{verbatim}

\begin{verbatim}
##  (Intercept)  sqft_living     sqft_lot    bathrooms        grade 
## 0.000000e+00 0.000000e+00 1.711092e-10 2.689854e-30 0.000000e+00
\end{verbatim}
\end{frame}

\begin{frame}[fragile]{Print Results}
\phantomsection\label{print-results-1}
\begin{verbatim}
## 
## Call:
## lm(formula = price ~ sqft_living + sqft_lot15 + bathrooms + bedrooms + 
##     grade, data = house_df, na.action = na.omit)
## 
## Coefficients:
## (Intercept)  sqft_living   sqft_lot15    bathrooms     bedrooms        grade  
##  -4.658e+05    2.341e+02   -7.113e-01   -2.894e+04   -4.161e+04    9.527e+04
\end{verbatim}

\begin{verbatim}
## 
## Call:
## lm(formula = price ~ poly(sqft_living, 2) + sqft_lot15 + bathrooms + 
##     bedrooms + grade, data = house_df)
## 
## Coefficients:
##           (Intercept)  poly(sqft_living, 2)1  poly(sqft_living, 2)2  
##            -1.824e+05              2.696e+07              1.148e+07  
##            sqft_lot15              bathrooms               bedrooms  
##            -7.345e-01             -1.565e+04             -1.723e+04  
##                 grade  
##             1.075e+05
\end{verbatim}
\end{frame}

\end{document}
