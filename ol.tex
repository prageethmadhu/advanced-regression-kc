%% Semplice beamer conforme al powerpoint ufficiale
%% dal sito di Ca' Foscari. Si basa sul tema "default"
%% mandate Modifiche e migliorie! Guido.Caldarelli@unive.it 
% Elenco Contributori 
% Guido Caldarelli, Matteo Brilli 

%\documentclass{beamer}
% decide below the aspect ratio between 16:9 and 4:3
%\documentclass[aspectratio=43]{beamer}
\documentclass[aspectratio=169]{beamer}

\usepackage[utf8]{inputenc}


% Questo tema commentato di sotto produce un beamer più tradizionale 
%\usetheme[secheader]{Boadilla}


%%%-----------------------------------------------------------%
%% Cambia colori da thema default
%% Questi sono i due colori ufficiali rosso e grigio
\definecolor {cfred}{rgb}{0.709,0.196,0.329} 	%{ 181 ,50 ,84}
\definecolor {cfgrey}{rgb}{0.537,0.537,0.537} 	%{ 137,137,137}
\definecolor {cflink}{rgb}{0.615,0.615,0.607} 	%{157,157,155}

\setbeamercolor{palette primary}{bg=cfred,fg=white}
\setbeamercolor{palette secondary}{bg=cfred,fg=white}
\setbeamercolor{palette tertiary}{bg=cfred,fg=white}
\setbeamercolor{palette quaternary}{bg=cfred,fg=white}
\setbeamercolor{structure}{fg=cfred}		 % itemize, enumerate, etc
\setbeamercolor{section in toc}{fg=cfred} 		 % TOC sections
% Override palette coloring with secondary
\setbeamercolor{subsection in head/foot}{bg=cfgrey,fg=white}
%%%------------------------------------------------------------

%% Definisce il blocco con riquadro che non è presente nel tema default (commentare se si usano altri temi)
\setbeamercolor{uppercolor}{fg=white,bg=cfred}%
\setbeamercolor{lowercolor}{fg=black,bg=white}%
\def \bblock{\begin{beamerboxesrounded}[upper=uppercolor,lower=lowercolor,shadow=true]}
\def \eblock{\end{beamerboxesrounded}}
%%-----------------------------------------------------------

%% Intestazione ripetuta per ogni slide
\addtobeamertemplate{headline}{%
\vspace{0.25cm} \ \ 
\includegraphics[height=1.0cm]{logobeamEN.png} 	% sostituire con logobeamIT.png per italiano
\hspace{0.641\textwidth}{\color{cflink} {\small www.unive.it}} %per 16:9
%\hspace{0.551\textwidth}{\color{cflink} {\small www.unive.it}} %per 4::3

\vspace{0.25cm}
{\color{cfred} \hrule \hrule  }
\textbf{}
}{}
%%-------------------------------------------------------------

%This block of code defines the information to appear in the Title page
%%%
\title[About Beamer] %optional
{About the Beamer class for Ca'Foscari researchers}\subtitle{A short story}

\author[Newton, Darwin] % (optional)
{I.~Newton\inst{1} \and C.~Darwin\inst{2}}

\institute[VFU] % (optional)
{
  \inst{1}%
  Department of Molecular Sciences and Nanosystems\\
  Ca' Foscari University, Via Torino 155, 30170 Venezia Mestre, Italy
  \and
  \inst{2}%
  European Centre for Living Technologies (ECLT)\\
 Ca' Bottacin, 3911 Dorsoduro Calle Crosera, 30123 Venice, Italy
 }

\date[VLC 2014] % (optional)
{Very Large Conference, April 2020}

\setbeamertemplate{frametitle}[default][right, rightskip=.5cm] {}
\addtobeamertemplate{frametitle}{\vspace*{-1.4cm}}{}
%End of title page configuration block
%------------------------------------------------------------


%------------------------------------------------------------
%The next block of commands puts the table of contents at the 
%beginning of each section and highlights the current section:

\AtBeginSection[]
{
  \begin{frame}
    \frametitle{Table of Contents}
    \tableofcontents[currentsection]
  \end{frame}
}
%------------------------------------------------------------

\begin{document}

%The next statement creates the title page.
\frame{\titlepage}
%---------------------------------------------------------
%This block of code is for the table of contents after
%the title page
\begin{frame}
\frametitle{Table of Contents}
\tableofcontents
\end{frame}
%---------------------------------------------------------


\section{First section}


%---------------------------------------------------------
%Changing visibility of the text
\begin{frame}
\frametitle{Sample frame title}
This is a text in second frame. For the sake of showing an example.

\begin{itemize}
    \item<1-> Text visible on slide 1
    \item<2-> Text visible on slide 2
    \item<3> Text visible on slides 3
    \item<4-> Text visible on slide 4
\end{itemize}
\end{frame}

%---------------------------------------------------------


%---------------------------------------------------------
%Example of the \pause command
\begin{frame}
In this slide \pause

the text will be partially visible \pause

And finally everything will be there
\end{frame}
%---------------------------------------------------------

\section{Second section}

%---------------------------------------------------------
%Highlighting text
\begin{frame}
\frametitle{Sample frame title}

In this slide, some important text will be
\alert{highlighted} because it's important.
Please, don't abuse it.

\begin{block}{Remark}
Sample text
\end{block}

\begin{alertblock}{Important theorem}
Sample text in red box
\end{alertblock}

\begin{examples}
Sample text in green box. The title of the block is ``Examples".
\end{examples}

\bblock{Custom defined block}
The theme does not use this kind of block; if needed use the newcommand ``bblock/eblock''
\eblock

\end{frame}
%---------------------------------------------------------


%---------------------------------------------------------
%Two columns
\begin{frame}
\frametitle{Two-column slide}

\begin{columns}

\column{0.5\textwidth}
This is a text in first column.
$$E=mc^2$$
\begin{itemize}
\item First item
\item Second item
\end{itemize}

\column{0.5\textwidth}
This text will be in the second column
and on a second tought this is a nice looking
layout in some cases.
\end{columns}
\end{frame}
%---------------------------------------------------------


\end{document}