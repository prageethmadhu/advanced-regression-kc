\documentclass{beamer}
\usepackage{graphicx}
\usepackage{booktabs}
\usepackage{hyperref}

\title{Regression and Prediction}
\author{Presented by ChatGPT}
\date{\today}

\begin{document}

\frame{\titlepage}

\begin{frame}{1. Regression and Supervised Learning}
    \begin{itemize}
        \item Regression examines relationships between predictor variables (X) and response variable (Y).
        \item Supervised learning: Training a model using known outcomes.
        \item Prediction vs. Explanation:
        \begin{itemize}
            \item Prediction: Estimating unknown Y values.
            \item Explanation: Understanding relationships between X and Y.
        \end{itemize}
    \end{itemize}
\end{frame}

\begin{frame}{2. Simple Linear Regression}
    \begin{itemize}
        \item Models a relationship between one predictor and one response variable: \( Y = b_0 + b_1X \)
        \item Key terms:
        \begin{itemize}
            \item Response variable (Y): Outcome being predicted.
            \item Predictor variable (X): Independent input variable.
            \item Regression coefficient (b₁): Measures impact of X on Y.
            \item Intercept (b₀): Value of Y when X = 0.
            \item Residuals: Difference between actual and predicted values.
        \end{itemize}
        \item Estimation method: Ordinary Least Squares (OLS).
    \end{itemize}
\end{frame}

\begin{frame}{3. Multiple Linear Regression}
    \begin{itemize}
        \item Expands simple regression to multiple predictor variables:
        \[ Y = b_0 + b_1X_1 + b_2X_2 + \dots + b_pX_p + e \]
        \item Key evaluation metrics:
        \begin{itemize}
            \item Root Mean Squared Error (RMSE), Residual Standard Error (RSE).
            \item R² (Coefficient of Determination) & Adjusted R².
            \item t-statistic & p-value to determine predictor significance.
        \end{itemize}
    \end{itemize}
\end{frame}

\begin{frame}{4. Model Selection and Feature Selection}
    \begin{itemize}
        \item Stepwise Regression: Backward elimination, forward selection.
        \item Akaike Information Criterion (AIC) & Bayesian Information Criterion (BIC).
        \item Cross-validation (k-fold): Splits data into multiple train-test sets.
    \end{itemize}
\end{frame}

\begin{frame}{5. Weighted Regression}
    \begin{itemize}
        \item Assigns different weights to observations based on importance.
        \item Used for inverse-variance weighting and survey analysis.
    \end{itemize}
\end{frame}

\begin{frame}{6. Factor Variables and Encoding Categorical Data}
    \begin{itemize}
        \item Factor Variables: Categorical variables needing transformation.
        \item Dummy Variables (One-Hot Encoding) & Reference Coding.
        \item Handling too many categories (e.g., zip codes) via grouping.
    \end{itemize}
\end{frame}

\begin{frame}{7. Model Interpretation Challenges}
    \begin{itemize}
        \item Collinearity: Highly correlated predictors affect coefficient interpretation.
        \item Confounding Variables: Missing key predictors lead to misleading relationships.
        \item Significance Testing: p-values, t-statistics to determine variable importance.
    \end{itemize}
\end{frame}

\begin{frame}{8. Regression Diagnostics and Model Validation}
    \begin{itemize}
        \item Residual Analysis: Detecting patterns in errors.
        \item Heteroskedasticity: Uneven variance in residuals.
        \item Outliers & Influential Points: Data points that impact model fit.
        \item Leverage (Hat Values): Influence of individual data points.
    \end{itemize}
\end{frame}

\begin{frame}{9. Regularization Techniques}
    \begin{itemize}
        \item Ridge Regression (L2): Shrinks regression coefficients.
        \item Lasso Regression (L1): Sets some coefficients to zero.
        \item Elastic Net: Combination of Ridge and Lasso.
    \end{itemize}
\end{frame}

\begin{frame}{10. Time Series and Regression}
    \begin{itemize}
        \item Autoregressive Models (AR), ARIMA for forecasting.
        \item Use of regression for time-dependent predictions.
    \end{itemize}
\end{frame}

\begin{frame}{11. Case Study: House Price Prediction}
    \begin{itemize}
        \item Predicting house sales price using features like:
        \begin{itemize}
            \item Square footage, bedrooms, location (zip code).
            \item Property type, building grade, number of bathrooms.
        \end{itemize}
        \item Challenges: Handling missing data, feature selection, avoiding overfitting.
    \end{itemize}
\end{frame}

\begin{frame}{12. Best Practices in Regression Modeling}
    \begin{itemize}
        \item Avoid overfitting using cross-validation.
        \item Check for collinearity among predictors.
        \item Use domain knowledge for meaningful features.
        \item Validate model assumptions via residual plots.
        \item Interpret coefficients carefully in presence of correlations.
    \end{itemize}
\end{frame}

\begin{frame}{Conclusion}
    \begin{itemize}
        \item Regression is essential for prediction and explanation.
        \item Proper feature selection and model validation improve reliability.
        \item Regularization prevents overfitting in complex models.
        \item Interpretation challenges require careful handling of correlations.
    \end{itemize}
\end{frame}

\end{document}
